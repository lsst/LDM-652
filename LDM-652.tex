\documentclass[DM,lsstdraft,STS,toc]{lsstdoc}
\usepackage{enumitem}
\usepackage{booktabs}
\usepackage{arydshln}

\input meta.tex

\setcounter{tocdepth}{2}
\begin{document}

\providecommand{\tightlist}{%
  \setlength{\itemsep}{0pt}\setlength{\parskip}{0pt}}

\def\product{LSST Science Platform}

\setDocCompact{true}

\title[Test Spec for \product]{\product~Design Review Charge}

\author{Leanne~P.~Guy}
\setDocRef{\lsstDocType-\lsstDocNum}
\setDocDate{\vcsdate}

\setDocAbstract {
This document provides the charge to the review committee for the \product{} (LSP) Design Review.
The review will be a formal and internal review of the planned LSP capabilities in the LSST operations era.
}

% Most recent last
\setDocChangeRecord{%
	\addtohist{0.1}{2018-10-16}{Initial charge}{Leanne Guy}
	\addtohist{0.2}{2018-10-25}{Updates to scope and charge questions following review at the LSP technical workshop}{Leanne Guy}
}

\setDocCurator{Leanne Guy}
\setDocUpstreamLocation{\url{https://github.com/lsst/ldm-652}}
\setDocUpstreamVersion{\vcsrevision}

\maketitle


% ADD CONTENT HERE

\section{Introduction}
\subsection{Objectives}
\label{sec:objectives}

The LSST Science Platform is a unified set of web applications and services made available to the scientific community to access, visualize, and perform 'next-to-the-data'
analysis of the LSST data.
The platform exposes the LSST data and services to the user through three primary user facing
``Aspects'' --- the web Portal, the JupyterLab analysis environment, and a machine-accessible Web API interface,
each providing  different ways to access the data  and analysis services provided by the LSST Data Access Centers (DACs).

Although primarily conceived as a platform for scientific analysis of the LSST data in the operations-era, the LSP will also be the major platform for integration and test activities during LSST commissioning,
and as such, the relevant stakeholders comprise not only members of the scientific community but also the LSST Camera and Commissioning teams.

The objective of this review is to evaluate the vision and design of the LSP against the LSST science requirements, and to verify both the current implementation
and the design for future planned operations-era aspects of the system.  LSST Data Management want to explore how the LSP can best meet the needs of the LSST science user community by
identifying issues now before committing to further implementation or reprioritization of already-planned work.


% Scope of the review
\subsection{Scope of the Review}
\label{sec:scope}
This is a design review of the LSST Science Platform (LSP), as defined and described in \citeds{LSE-319}: LSST Science Platform Vision Document.
The services provided by the LSP can be decomposed into the three different ``Aspects'':
\begin{itemize}
\item A web Portal designed to provide essential data access and visualization services through a simple-to-use website.
\item A JupyterLab environment, that will provide a Jupyter Notebook-like interface enabling next-to-the-data analysis.
\item An extensive set of Web APIs that the users will be able to use to remotely examine the LSST data with set familiar tools.
\end{itemize}

The LSST teams developing the Science Platform --- Science User Interface and Tools (SUIT) at IPAC, SQuaRE at LSST-Tucson, and DAX at SLAC --- are
requested to demonstrate the detailed product design to be realized as well as the realization process.

Science use cases addressing all four of the LSST key science themes - probing dark energy and dark matter,
taking an inventory of the Solar System, exploring the transient optical sky, and mapping the Milky Way --- should be addressed.
These use cases should cover database-oriented data access, data access for large-scale analytics (possibly non-database), and user interfaces. 
\newtext{Note that the LSST database, QServ, is considered to be outside the scope of this review,  however certain aspects of the database design can effect the  performance of the LSP. Where pertinent, those aspects of the QServ design relevant to science users and the ability of the LSP to satisfy LSST science requirements should be presented. }

A representative group of LSST stakeholders will be asked to provide science-oriented feedback and recommendations on the design of the LSST Science Platform.

\subsection{Review Materials}
\label{sec:materials}

Materials to be presented for each Aspect of the LSP will include:
\begin{itemize}
\item Design documents covering requirements, architecture and interfaces \newtext{including details pertaining to the underlying database design that could impact the science performance of the LSP;}
\item Current status of development including technical progress;
\item Test specifications, verification and validation plans;
\item Description of future development plans and milestones to completion and their relationship to major project milestones though commissioning and the start of operations; and
\item Risk and mitigation plans.
\end{itemize}

All relevant documentation should be provided to the review committee no later than 2 weeks prior to the date of the review.

\subsection{Applicable Documents}
\label{sec:docs}

\addtocounter{table}{-1}

\begin{tabular}[htb]{l l}
\citeds{LSE-319} & LSST Science Platform Vision Document \\
\citeds{LDM-554} & LSST Science Platform Requirements\\
\citeds{LDM-542} & LSST Science Platform Design\\
\citeds{LSE-163} & LSST Data Products Definition Document \\
\citeds{LPM-17} & LSST Science Requirements Document (SRD) \\
\citeds{LSE-61}  & LSST DM Subsystem Requirements (DMSR)  \\
\citeds{LDM-148} & LSST Data Management System Design \\
\citeds{LSE-30} & LSST Observatory System Specifications (OSS) \\
\end{tabular}

% references
\subsection{References\label{sect:references}}
\renewcommand{\refname}{}
\bibliography{lsst,refs,books,refs_ads}

%acronyms
\subsection{Acronyms \label{sect:acronyms}} % include acronyms.tex generated by the generateAcronyms.py (in texmf/scripts)
\addtocounter{table}{-1}
\begin{longtable}{|l|p{0.8\textwidth}|}\hline
\textbf{Acronym} & \textbf{Description}  \\\hline

DM & Data Management \\\hline
DMSR & DM System Requirements \\\hline
LDM & LSST Data Management \\\hline
LPM & LSST Project Management (Document Handle) \\\hline
LSE & LSST Systems Engineering (Document Handle) \\\hline
LSP & LSST Science Platform \\\hline
LSST & Large Synoptic Survey Telescope \\\hline
OSS & Observatory System Specifications \\\hline
SRD & Software Requirements Document \\\hline
STS & System Test Specification \\\hline
SUIT & Science User Interface and Tools \\\hline
s & second; SI unit of time \\\hline
\end{longtable}



\newpage
%%% Charge questions
\section{Charge to the review committee}
\label{sec:charge}
The review committee is asked to assess the following items based on the material presented and made available.
%
\begin{enumerate}
% We need to do some work before the review to get this into shape. Our requirements traceability is not great. Would propose to get the key people together and knock it off in a dedicated session
\item Is the traceability of requirements from higher design documents, e.g. from \citeds{LPM-17}:The LSST Science Requirements Document,
to \citeds{LDM-554}:The LSST Science Platform Requirements, complete and will it ensure coverage of the four key LSST science themes?
%
\item Are the stakeholders clearly identified and understood? Have the requirements been prioritized and communicated to the stakeholders?
%
% Portal in LDM-554 are the portal requirements. DAX reqs less clear - DMTN-090 - more VO centric design. - Well captured.
\item Does the design presented in \citeds{LDM-542}: LSST Science Platform Design capture the requirements for the LSP as detailed in \citeds{LDM-554}:The LSST Science Platform Requirements?
%
\item Are the verification, validation and software quality assurance plans adequate?
%
\item Does the performance of the current system and its development status inspire confidence that both the interim and operations-era functionality can be delivered?
%
\item How does the design of the LSP compare with that of other contemporary astronomical data archives and interfaces, or, more generally, other scientific data analysis environments? How well does the design and current version reflect trends in software engineering? Do the current design and technology choices give confidence that the LSP can evolve over time with the needs of 21st century astronomy?
%
% Removed 'that have not been addressed' following presentation at LSP workshop
\item \newtext{Are there items of significance in the design that would limit the science harvest of LSST?}
%
% No substantive risk analysis has been done since the LSP concept was finalized (some Portal risks had been recognized early on). The risk register is not very well matched and not matched to what we ended up building.
% GPDF should write a note on risks associated with the LSP and pass to  Wil to assimilate.
\item Are the risks associated with the design of the LSST Science Platform understood and adequately captured? Are there any overlooked areas of risk?
%
% Note that in the event that  DM descope-10 is enacted prior to this review, this charge question will be reworded to be more specific,  acknowledging that a descope has occured. 
% The question as to whether the remaining LSP is sufficient will be posed. 
\item If cuts had to be made, what areas could be de-scoped with minimal implications for LSST science? What is the scope for use of third party-tooling in place of in-house development? 
({\color{red}{Note: If DM-10 is invoked, or if the variant to freeze rather than eliminate the Portal is invoked, this question will be rephrased to reflect accordingly.}})

%\color{red}{(Alternative if DM-10 is approved) In light of the recent implementation of de-scope item DM10: Eliminate the Portal Aspect of the Science Platform, are the the remaining two aspects of the LSP sufficient to 
%ensure LSST science harvest covering the four key science themes? What, if any, third party-tooling could be substituted for the Portal aspect to ensure that the LSP can deliver the science promise of LSST?}
\end{enumerate}

In addition, the committee is asked to provide \newtext{actionable advice on addressing any issues raised during the course of the review as well as guidance based on experience that will ensure the success of the LSST Science Platform.}

%%% The committee
\section{Review Committee Composition}
The review committee shall be composed of persons external to LSST Data Management
and who are representative of the LSST DM stakeholders and principal users of the LSST Science Platform.

% Note stakeholders comprise: LSST science collaborations, cross-subsystem integration, commissioning, early preparation by science collaborations

%%% Committee Report
\section{Committee report}
At the conclusion of the review, the committee is requested to provide a written report  to the LSST Project Scientist within 2 weeks  detailing
their major recommendations.  The report should contain the consensus of the committee's findings, comments and recommendations. The committee is also requested to
provide a verbal exit briefing based around a closeout presentation on the last day to convey actions and comments. This closeout briefing need not
necessarily be final and it is understood that the committee can revise their findings prior to presentation in the final written report.

%%% Proposed agenda
% Said we would do it here: March ? In the Spring. March/ early April. Before May IVOA meeting.
\section{Proposed Agenda}
We propose a full 2-day review, possibly with one full day and two half days either side in Spring 2019, possibly in March or April.
The review committee are requested to be present for the full duration.

(Draft) Proposed Review Agenda:

\textbf{Wednesday 2019-02-20} \\
\begin{tabular}[htb]{l l}
9:00 - 12:30 & Review \\
12:30 - 13:30 & Lunch \\
13:30 - 17:00 & Review \\
\end{tabular}

\textbf{Thursday 2019-02-21} \\
\begin{tabular}[htb]{l l}
9:00 - 12:30 & Review \\
12:30 - 13:30 & Lunch \\
13:30 - 17:00 & Review \\
\end{tabular}

\newtext{
% Allow time for the committee to convene to
\textbf{Friday 2019-02-22} \\
\begin{tabular}[htb]{l l}
9:00 - 11:30 &Committee private session \\
11:30- 12:30 & Closeout session\\
\end{tabular}
}	

\clearpage

\end{document}
